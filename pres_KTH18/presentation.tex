\documentclass{beamer}

\beamertemplatenavigationsymbolsempty

\usepackage{etex}
% packages required  by CVMono
\usepackage{mathptmx}
\usepackage{helvet}
\usepackage{courier}
\usepackage{type1cm}
\usepackage{makeidx}
\usepackage{graphicx}
\usepackage{multicol}
\usepackage[bottom]{footmisc}
\usepackage{placeins}



%%%
\usepackage[UTF8x]{inputenc}
\usepackage[german,english]{babel}
\usepackage{amsmath}
\usepackage{amssymb}
%\usepackage{amsthm}
\usepackage{pifont}% for checkmarks \ding{51}
\makeatletter
\@ifclassloaded{beamer}{}{\usepackage{paralist}}
\makeatother
\usepackage{float}
\usepackage{adjustbox}

\usepackage[%
    font={small,sf},
    labelfont=bf,
    format=hang,    
    format=plain,
    margin=0pt,
    width=0.8\textwidth,
]{caption}
\usepackage[list=true]{subcaption}
%\usepackage{alltt}

% symbols for the naturals (\mathbbm{N}), integers (\mathbbm{Z}), etc.
\usepackage{bbm}

% nice tables (read the accompanying docs!)
\usepackage{booktabs}
\usepackage{hyperref}% for bookmarks in PDF file
\usepackage{url}%showing url in bibtex
% source code and diagrams
\usepackage{tikz}
\usepackage{tikz-timing}

% code and monostace
\usepackage{listings}
\usepackage{fancyvrb}
\usepackage{fixltx2e}
\usepackage{textcomp}

\usepackage{datetime}



\usepackage{tikz}
\usepackage{import}


% inference rules
\usepackage{mathpartir}

\let\proof\relax
\let\endproof\relax
\usepackage{amsthm}

% set literals, \Set{ ... | ... }, \Set{ ..., ..., ..., ... }
\usepackage{braket}

% interval literals, \interval{...}{...}, \interval[open left]{...}{...} etc
\usepackage{interval}
\intervalconfig{
	soft  open  fences,
	separator symbol=:,
}

\usepackage{import}
\usepackage{bm}

% \centernot : like \not, but better alignment
\usepackage{centernot}

% has nice things like \cmark and \xmark
\usepackage{pifont}
\usepackage{wasysym}


\usepackage{multirow}
\usepackage{nameref}

% has multiple references grouped together in clever ways; \cref
\usepackage[capitalise]{cleveref}

% manipulates strings
\usepackage{xstring}
%\theoremstyle{definition}
%\newtheorem{definition}{Definition}
%\newtheorem{lemma}{Lemma}
%\newtheorem{theorem}{Theorem}

% note: the next 2 environments are used only in sysbook; have to be adapted wrt to the style reference
\newtheorem{invariant}{Invariant}
\newtheorem{ccond}{Context Condition}

% make float placement on top the default:
\floatplacement{figure}{t}
\floatplacement{table}{t}
\renewcommand{\floatpagefraction}{0.8}

% tikz
\tikzset{timing/slope=0.1}
\tikzset{timing/.style = {y=2.5ex, x = 4.0ex}}

% other
\graphicspath{{figures/}}
\bibliographystyle{alpha}

% remove later
\newcommand{\typo}[1]{\textcolor{blue}{\textbf{#1}}}
\newcommand{\bbox}[1]{\fcolorbox{blue}{white}{#1}}

%%%%%%%%%%%%%%%%%%%%%%%%%%%%%%%%%%%%%%%%%%%%%%%%%%%%%%%%%%%%%%%%
% Define your desired shortcuts here:
\mathchardef\hy="2D
\newcommand{\cm}{\mbox{,}}
\newcommand{\setB}{\mathbb{B}}
\newcommand{\setN}{\mathbb{N}}
\newcommand{\setZ}{\mathbb{Z}}
\newcommand{\setR}{\mathbb{R}}
\newcommand{\setS}{\mathbb{S}}
\newcommand{\B}{\mathbb{B}}
\newcommand{\N}{\mathbb{N}}
\newcommand{\R}{\mathbb{R}}
\newcommand{\nat}{\mathbb{N}}
\newcommand{\Z}{\mathbb{Z}}
\newcommand{\NOT}[1]{\overline{#1}}
\newcommand{\AND}{\land}
\newcommand{\OR}{\lor}
\newcommand{\Sum}{\sum\limits}
\newcommand{\tmod}{\mathrel{\mathrm{tmod}}}
\newcommand{\pskip}{\smallskip\par\noindent}
\newcommand{\ignore}[1]{\relax}
\newcommand{\figscale}{0.834}
%\renewcommand{\pskip}{\smallskip\bbox{\mbox{\quad}}\par\noindent}
\newcommand{\smalltt}[1]{\text{\small\texttt{#1}}}
\newcommand{\super}[1]{\textsuperscript{#1}}
\newcommand{\sub}[1]{\textsubscript{#1}}
\newcommand{\langlett}{{\fontfamily{cmbr}\selectfont\textlangle}}
\newcommand{\ranglett}{{\fontfamily{cmbr}\selectfont\textrangle}}
\renewcommand{\theFancyVerbLine}{\text{\small\arabic{FancyVerbLine}:}}
\renewenvironment{verbatim}[0]{\Verbatim}{\endVerbatim}

% figures
\newcommand{\tfig}[3]{%
\centering%
\adjustbox{scale = \figscale}{\input{figures/#1.pdf_t}}%
\caption{#3}\label{fig:#2}%
}

\newcommand{\tsubfigV}[3]{%
\subcaptionbox{\label{#2} #3}{\import{figures/}{#1.pdf_tex}}%
}

\newcommand{\includefig}[1]{\import{figures/}{#1.pdf_tex}}


\newcommand{\tschemfig}[4]{%
\figure[tp]%
\addtolength{\subfigcapskip}{0.1in}%
\centering%
\subfigure[\label{fig:#1} symbol]{\trimbox{-0.5\textwidth+0.5\width 0pt 0pt}{\raisebox{#3mm}{\adjustbox{scale = \figscale}{\input{figures/#1.pdf_t}}}}}\\%
\subfigure[\label{fig:#1impl} implementation]{\trimbox{-0.5\textwidth+0.5\width 0pt 0pt}{\raisebox{#4mm}{\adjustbox{scale = \figscale}{\input{figures/#1impl.pdf_t}}}}}%
\caption{#2}\label{fig:#1-all}%
\endfigure%
}

%note: command "remark" is already defined in the style file
%\newcommand{\remark}[1]{\marginpar{\framebox{\parbox[t]{16mm}{
  %\tiny\raggedright #1}}}}
%\newcommand{\oldremark}[1]{\relax}

% code
\fvset{
	frame=lines,
	framerule=0.2pt,
	framesep=4pt,
	resetmargins=true,
	xleftmargin=16pt,
	xrightmargin=16pt,
%	numbers=left,
%	numberblanklines=false,
%	numbersep=-16pt,
	tabsize=0,
	fontfamily=courier,
	fontsize=\small,
	commandchars=\\\@\#
}

\lstnewenvironment{code}
    {\lstset{}%
      \csname lst@SetFirstLabel\endcsname}
    {\csname lst@SaveFirstLabel\endcsname}
    \lstset{
      language=Haskell,
      mathescape = true,
      commentstyle= \sffamily\itshape,
      basicstyle=\small\ttfamily,
      keywordstyle=\color{DarkBlue}\bfseries,
      flexiblecolumns=false,
      basewidth={0.5em,0.45em},
      morecomment=[l]{//},
      numbers = left,
      numberstyle = \color{Gray}\tiny\ttfamily,
      breaklines = true,
      literate={+}{{$+$}}1 {*}{{$*$}}1 {=}{{$=$}}1 {!=}{{$\neq$}}2
               {>}{{$>$}}1 {<}{{$<$}}1 {\\}{{$\lambda$}}1
               {->}{{$\rightarrow$}}2 {>=}{{$\geq$}}2 {<-}{{$\leftarrow$}}2
               {<=}{{$\leq$}}2 {=>}{{$\Rightarrow$}}2
               {>>}{{$\gg$}}1
               {<<}{{$\ll$}}1
               {|}{{$\mid$}}1
               {||}{{$\vee$}}2 {&&}{{$\wedge$}}2
               {@}{{$\circ$}}1
               {?}{{{\color{DarkBlue}{?}}}}1
               {...}{{$\ldots$}}3
    }

% forbids footnotes to travel
%\interfootnotelinepenalty=1000000000
\interfootnotelinepenalty=10000

% theorems of amsthm

% config of the tikz package
\usetikzlibrary{automata,positioning}

%new shorthands

% Tombstone at end of proof (amsthm does this now)
%\renewcommand\endproof{~\hfill\qed}

% std mathcal font
\DeclareMathAlphabet{\mathcal}{OMS}{cmsy}{m}{n}

% new utils
\newcommand{\tikscale}{1.0}

\newcommand{\tbl}[3]{%
\caption{#3}%
\label{tab:#2}%
\centering%
\input{tables/#1}%
}

\newcommand{\tpic}[3]{%
\centering%
\adjustbox{scale = \tikscale}{%
\tikzpicture[%
	>=stealth,%
	shorten >=0.5pt,%
	node distance=2.5cm,%
	on grid,%
	auto,%
	every node/.append style={very thin},%
	every path/.append style={very thin}%
]%
\input{drawings/#1}%
\endtikzpicture}%
\caption{#3}\label{fig:#2}%
}

\newcommand{\tfigscaled}[4]{%
\centering%
\adjustbox{scale = #4}{\input{figures/#1.pdf_t}}%
\caption{#3}\label{fig:#2}%
}

%%% REDUCTION PROOFS
%% units making steps
\newcommand{\MMU}{\mathrm{MMU}}
\newcommand{\SB}{\mathrm{SB}}
\newcommand{\APIC}{\mathrm{APIC}}
\newcommand{\IOAPIC}{\mathrm{IOAPIC}}
\newcommand{\DEV}{\mathrm{DEV}}

%% classes of schedules
\newcommand{\LSBS}{\mathrm{LSBS}}
\newcommand{\ABS}{\mathrm{ABS}}
\newcommand{\REDUCIBLE}{\mathrm{RED}}
\newcommand{\IND}{\mathrm{ORD}}

\newcommand{\IBLOCK}{\mathrm{IBLOCK}}
\newcommand{\CBLOCK}{\mathrm{CBLOCK}}
\newcommand{\EIBLOCK}{\mathrm{EIBLOCK}}
%% 
\newcommand{\sbcomp}[1]{{\overline{#1}}}


%% Makes it into a single word...
\newcommand{\inputsafe}{in\hy safe}
%% communication relations
\DeclareMathOperator{\fwdsync}\vartriangleright
\DeclareMathOperator{\rsync}{{\ooalign{$\vartriangleright$\cr\kern0.04em
  $\ast$\cr}}}
\DeclareMathOperator{\sync}{\blacktriangleright}
\DeclareMathOperator{\talks}{\sync\!\!\!\!\!\fwdsync}
\DeclareMathOperator{\rtalks}{\sync\!\!\!\!\!\rsync}
\DeclareMathOperator{\backsync}{\vartriangleleft}
\DeclareMathOperator{\leaksync}{\blacktriangleright_2}

%% memory update 
\DeclareMathOperator*{\updatedby}{\circledast}
\DeclareMathOperator*{\uninfupdatedby}{\circledcirc}


%% function restriction
\newcommand\restr[2]{{% we make the whole thing an ordinary symbol
  \left.\kern-\nulldelimiterspace % automatically resize the bar with \right
  #1 % the function
  \vphantom{\big|} % pretend it's a little taller at normal size
  \right|_{#2} % this is the delimiter
  }}
%% intersects
\DeclareMathOperator*{\intersects}{\dot\cap}
%% List comprehension, taken from braket
{\catcode`\|=\active
  \xdef\List{\protect\expandafter\noexpand\csname List \endcsname}
  \expandafter\gdef\csname List \endcsname#1{\left[%
     \ifx\SavedDoubleVert\relax \let\SavedDoubleVert\|\fi
     \:{\let\|\SetDoubleVert
     \mathcode`\|32768\let|\SetVert
     #1}\:\right]}
}

%% absolute/length
\providecommand{\abs}[1]{\lvert#1\rvert}
\providecommand{\len}[1]{\lvert#1\rvert}

%% from
\newcommand{\from}\leftarrow


%% Aliases for readablity
\newcommand{\Iota}I

%% checkmark and x
\newcommand{\cmark}{\ding{51}}%
\newcommand{\xmark}{\ding{55}}%

%% When the machines agree
\providecommand{\MI}{\ast}
\providecommand{\M}[1]{\MI} %can be replaced by #1 in case we decide to distinguish the machine types after all



%normal red is too bright
\definecolor{red}{HTML}{800000} 

\usetikzlibrary{matrix,shapes,arrows,fit,tikzmark}
%\usepackage{bm}


% \C is used inside figures created by inkscape, which sadly does not have good alignment support for tex.
\newcommand\C[1]{\raisebox{-0.5\height}{\texttt{#1}}} 

\title{Verifying Concurrent Operating Systems}
\author{Jonas Oberhauser}
\date{\today} % Date, can be changed to a custom date
\begin{document}
\begin{frame}
\titlepage % Print the title page as the first slide
\end{frame}


\begin{frame}
Sequential verification
\begin{center}
	\texttt{
		\begin{tabular}{l}
			i++; 
		\\ i++;
		\end{tabular}
	}
\end{center}

After execution do we have 
\[ \texttt{i} = 2 \ ? \]

Trivial
\end{frame}



\begin{frame}
In reality (e.g., on MIPS):
\begin{center}
	\texttt{
		\begin{tabular}{l}
			fetch;
			\\t1 = i;
			\\fetch;
			\\ t1 = t1 + 1; 
			\\fetch;
			\\ i = t1;
			\\
			\\ fetch;
			\\ t1 = i;
			\\ fetch;
			\\ t1 = t1 + 1; 
			\\ fetch;
			\\ i = t1;
		\end{tabular}
	}
\end{center}

After execution do we have 
\[ \texttt{i} = 2 \ ? \]

Equally Trivial
\end{frame}


\begin{frame}
Multiple threads, shared state

\begin{center}
	\texttt{
		\begin{tabular}{l||l}
			i++; & i++;
		\end{tabular}
	}
\end{center}

After execution do we have 
\[ \texttt{i} = 2 \ ? \]

\end{frame}


\begin{frame}
Interleavings:

\begin{center}
	\scalebox{.6}{
		\begin{tabular}{ccc}
			\includefig{increment_coarse12} & \hspace{4em} & \includefig{increment_coarse21} 
		\end{tabular}}
\end{center}
\end{frame}


\begin{frame}
Break up into small steps:

\begin{center}
	\texttt{
		\begin{tabular}{l||l}
			t1 = i; & t2 = i;
		\\	i = t1+1; & i = t2+1;
		\end{tabular}
	}
\end{center}

\onslide<2->
Look at all interleavings:


\begin{center}
	\scalebox{.45}{
		\begin{tabular}{ccccc}
			\includefig{increment_schedule1122} & \hspace{4em} & \includefig{increment_schedule1212} & \hspace{4em} & \includefig{increment_schedule1221}\\
			\includefig{increment_schedule2211} & \hspace{4em} & \includefig{increment_schedule2121} & \hspace{4em} & \includefig{increment_schedule2112}	
	\end{tabular}}
\end{center}

New behavior:
\[ \texttt{i} = 1 \ \lor \ \texttt{i} = 2 \]
\end{frame}




\begin{frame}
Break up into smaller steps:

\begin{center}
	\texttt{
		\begin{tabular}{l||l}
			fetch; & fetch;
			\\ t1 = i; & t2 = i;
			\\ fetch; & fetch;
			\\ t1 = t1+1; & t2 = t2+1;
			\\ fetch; & fetch;
			\\ i = t1; & i = t2;
		\end{tabular}
	}
\end{center}

New behaviors?

Look at all 924 interleavings???
\onslide<2->
\begin{center}
	\scalebox{0.6}{\includegraphics{figures/nope-im-out}}
\end{center}
\end{frame}

\begin{frame}
Step 1 to a better life: \textbf{order reduction}

\begin{enumerate} 
	\item Answers the question: 
	
	\begin{center}
		``When did we break it up enough to get all possible behaviors?''
	\end{center}
	\item Theorem: if program satisfies some software conditions, breaking it down further will not add new behaviors
\end{enumerate}
\end{frame}



\begin{frame}
Breaking up can be modeled by cooperative vs preemptive interleaving

\begin{center}
	\texttt{
		\begin{tabular}{l||l}
			t1 = i, & t2 = i,
			\\	i = t1+1; & i = t2+1;
		\end{tabular}
	}
\end{center}

\begin{description}
	\item<2->[Cooperative scheduler] will only interleave at `;' (called \textbf{interleaving points} (IP))
	
	\begin{center}
		\scalebox{.3}{
			\begin{tabular}{ccc}
				\includefig{increment_schedule1122} & \hspace{4em} & \includefig{increment_schedule2211} \\
		\end{tabular}}
	\end{center}
	\item<3->[Preemptive scheduler] will interleave at `;' and `,'
	
	\begin{center}
		\scalebox{.3}{
			\begin{tabular}{ccccc}
				\includefig{increment_schedule1122} & \hspace{4em} & \includefig{increment_schedule1212} & \hspace{4em} & \includefig{increment_schedule1221}\\
				\includefig{increment_schedule2211} & \hspace{4em} & \includefig{increment_schedule2121} & \hspace{4em} & \includefig{increment_schedule2112}	
		\end{tabular}}
	\end{center}
\end{description}

\end{frame} 



\begin{frame}
\begin{center}
	\texttt{
		\begin{tabular}{l||l}
			t1 = i, & t2 = i,
			\\	i = t1+1; & i = t2+1;
		\end{tabular}
	}
\end{center}

Observation: 
\begin{enumerate} 
	\item \texttt{t1} only accessed by Thread 1
	\item \texttt{t2} only accessed by Thread 2
	\item \texttt{i} accessed by both, concurrently
\end{enumerate}

Make this formal with ownership
\end{frame} 

\begin{frame}
\alt<2->{\alt<3->{Introduce dynamic ownership:
		\begin{center}
			\includefig{MLayoutOwnedByTwo}
	\end{center}}{Introduce dynamic ownership:
		\begin{center}
			\includefig{MLayoutOwnedAnnotated}
\end{center}}}{Basic layout: processor registers and shared memory
	\begin{center}
		\includefig{MLayoutUnowned}
\end{center}}
\end{frame} 


\begin{frame}
\textbf{Software Condition 1)}: annotate accesses based on ownership as local or shared

\begin{center}
	\includefig{MLayoutOwnedAnnotations}
\end{center}

Shared access is \textbf{linearization point} (LP)
\end{frame} 

\begin{frame}
\textbf{Software Condition 2)}: at most one LP between two IPs

Violated:
\begin{center}
	\texttt{
		\begin{tabular}{l||l}
			t1 = \{i\}, & t2 = \{i\},
			\\	\{i\} = t1+1; & \{i\} = t2+1;
		\end{tabular}
	}
\end{center}

Satisfied:
\begin{center}
\texttt{
	\begin{tabular}{l||l}
		t1 = \{i\}; & t2 = \{i\};
		\\	\{i\} = t1+1; & \{i\} = t2+1;
	\end{tabular}
}
\end{center}
Satisfied:
\begin{center}
	\texttt{
		\begin{tabular}{l||l}
		t1 = \{i\}; & t2 = \{i\};
		\\ t1 = t1 + 1, & t2 = t2 + 1,
		\\	\{i\} = t1; & \{i\} = t2;
	\end{tabular}
}
\end{center}
\end{frame}


\begin{frame}
\textbf{Software Condition 3)}: After reaching LP, an IP must be reachable

Violated:
\begin{center}
\texttt{
	\begin{tabular}{l}
		\{i\} = 1,
		\\ while(1) \{
		\\ \};
	\end{tabular}
}
\end{center}
\end{frame} 


%%Taken from https://tex.stackexchange.com/questions/240542/adding-a-rectangular-box-with-tikz-to-table-beamer

	% Some options common to all the nodes and paths
	\tikzset{   
		every picture/.style={remember picture,baseline},
		every node/.style={anchor=base,align=center,outer sep=1.5pt},
		every path/.style={thick},
	}
	
	\newcommand\marktopleft[1]{%
		\tikz[overlay,remember picture] 
		\node (marker-#1-a) at (.4em,.1em) {};%
	}
	\newcommand\markbottomright[2]{%
		\tikz[overlay,remember picture] 
		\node (marker-#1-b) at (-.4em, .5em) {};%
		\tikz[overlay,remember picture,inner sep=3pt]
		\node[draw=#2,rounded corners,fit=(marker-#1-a.north west) (marker-#1-b.south east)] {};%
	}

%%end Taken

\begin{frame}[fragile,allowframebreaks]
\begin{center}
	\begin{tabular}{l||l} \texttt{fetch,} & \texttt{fetch,}
		\\ \texttt{t1 = \{i\},} \hspace{1em}  & \texttt{t2 = \{i\};} \hspace{1em} 
		\\ \texttt{fetch,} & \texttt{fetch,}
		\\ \texttt{t1 = t1 + 1;} & \texttt{t2 = t2 + 1,}
		\\ \texttt{fetch,} & \texttt{fetch,}
		\\	\texttt{\{i\} = t1;} \hspace{1em}  & \texttt{\{i\} = t2;} \hspace{1em} 
	\end{tabular}
\end{center}
\vspace{2.09375em}

\framebreak
\begin{center}
		\begin{tabular}{l||l} \marktopleft{a1}\texttt{fetch,} & \marktopleft{a2}\texttt{fetch,}
			\\ \texttt{t1 = \{i\},}  & \texttt{t2 = \{i\};} \hspace{1em} \markbottomright{a2}{red}
			\\ \texttt{fetch,} & \marktopleft{b2}\texttt{fetch,}
			\\ \texttt{t1 = t1 + 1;}\markbottomright{a1}{blue} & \texttt{t2 = t2 + 1,}
			\\ \marktopleft{b1}\texttt{fetch,} & \texttt{fetch,}
			\\	\texttt{\{i\} = t1;} \hspace{1.45em} \markbottomright{b1}{blue} & \texttt{\{i\} = t2;} \hspace{1em} \markbottomright{b2}{red}
		\end{tabular}
\end{center}


%\framebreak
%\begin{center}
%	\begin{tabular}{l||l} \marktopleft{a1}\texttt{t1 = \{i\};}  & \marktopleft{a2}\texttt{t2 = \{i\};}
%		\\ \hphantom{\texttt{t1 = \{i\};}} \hspace{1em} \markbottomright{a1}{blue} & \hphantom{\texttt{t2 = \{i\};}} \hspace{1em} \markbottomright{a2}{red}
%		\\[5pt] \marktopleft{b1}\texttt{\{i\} = t1;} & \marktopleft{b2}\texttt{\{i\} = t2;}
%		\\ \hphantom{\texttt{t1 = t1 + 1,}} & \hphantom{\texttt{t1 = t1 + 1,}}
%		\\
%		\\	\hphantom{\texttt{\{i\} = t1;}} \hspace{1em} \markbottomright{b1}{blue} & \hphantom{\texttt{\{i\} = t2;}} \hspace{1em} \markbottomright{b2}{red}
%	\end{tabular}
%\end{center}
\end{frame}
\begin{frame}
Particular execution (Let $\alpha_i, \beta_i$ be the addresses of the fetched instructions)
\begin{center}
	\begin{tabular}{ccl||lccc}
		O & P & & & O & P 
		\\ $\emptyset$ &  $\Set{\alpha_1}$ & \texttt{fetch ($\alpha_1$),} & & $\emptyset$ &  $\Set{\beta_1}$%
		\\ $\emptyset$ &  $\Set{\alpha_1}$ & \texttt{t1 = \{i\},} & & $\emptyset$ &  $\Set{\beta_1}$%
		\\ $\emptyset$ & $\Set{\alpha_2, \alpha_3}$ & \texttt{fetch  ($\alpha_2$),} & & $\emptyset$ &  $\Set{\beta_1}$%
		\\ $\emptyset$ & $\Set{\alpha_2, \alpha_3}$ & \texttt{t1 = t1 + 1;} & & $\emptyset$ &  $\Set{\beta_1}$%
		\onslide<2->{
		\\[4pt] $\emptyset$ &  $\Set{\alpha_2, \alpha_3}$ & & \texttt{fetch ($\beta_1$),} & $\emptyset$ &  $\Set{\beta_1}$%
		\\ $\emptyset$ &  $\Set{\alpha_2, \alpha_3}$ & & \texttt{t2 = \{i\};} & $\emptyset$ &  $\Set{\beta_1}$ &}
		\onslide<3->{
		\\[4pt] $\emptyset$ &  $\Set{\alpha_2, \alpha_3}$ & & \texttt{fetch ($\beta_2$),} & $\emptyset$ &  $\Set{\beta_2,\beta_3}$%
		\\ $\emptyset$ &  $\Set{\alpha_2, \alpha_3}$ & & \texttt{t2 = t2 + 1,} & $\emptyset$ &  $\Set{\beta_2,\beta_3}$%
		\\ $\emptyset$ &  $\Set{\alpha_2, \alpha_3}$ & & \texttt{fetch ($\beta_3$),} & $\emptyset$ &  $\Set{\beta_2,\beta_3}$%
		\\ $\emptyset$ &  $\Set{\alpha_2, \alpha_3}$ & & \texttt{\{i\} = t2;} & $\emptyset$ &  $\Set{\beta_2,\beta_3}$ &}
		\onslide<4->{
		\\[4pt] $\emptyset$ & $\Set{\alpha_2, \alpha_3}$ & \texttt{fetch ($\alpha_3$),} & & $\emptyset$ &  $\emptyset$%
		\\ $\emptyset$ & $\Set{\alpha_2, \alpha_3}$ & \texttt{\{i\} = t1;} & & $\emptyset$ &  $\emptyset$&}
	\end{tabular}
\end{center}
\end{frame}


\begin{frame}
Program obeys conditions in all \textbf{cooperative} executions
\begin{center}\scalebox{0.26}{
\begin{tabular}{ccccccc}
	\begin{tabular}{ccl||lccc}
		\\ $\emptyset$ & $\Set{\alpha_1}$ & \texttt{fetch ($\alpha_1$),} & & $\emptyset$ &  $\Set{\beta_1}$%
		\\ $\emptyset$ & $\Set{\alpha_1}$ & \texttt{t1 = \{i\},} & & $\emptyset$ &  $\Set{\beta_1}$%
		\\ $\emptyset$ & $\Set{\alpha_2, \alpha_3}$ & \texttt{fetch  ($\alpha_2$),} & & $\emptyset$ &  $\Set{\beta_1}$%
		\\ $\emptyset$ & $\Set{\alpha_2, \alpha_3}$ & \texttt{t1 = t1 + 1;} & & $\emptyset$ &  $\Set{\beta_1}$%
		\\[4pt] $\emptyset$ & $\Set{\alpha_2, \alpha_3}$ & \texttt{fetch ($\alpha_3$),} & & $\emptyset$ &  $\Set{\beta_1}$%
		\\ $\emptyset$ & $\Set{\alpha_2, \alpha_3}$ & \texttt{\{i\} = t1;} & & $\emptyset$ &  $\Set{\beta_1}$%
		\\[4pt] $\emptyset$ &  $\emptyset$ & & \texttt{fetch ($\beta_1$),} & $\emptyset$ &  $\Set{\beta_1}$%
		\\ $\emptyset$ &  $\emptyset$ & & \texttt{t2 = \{i\};} & $\emptyset$ &  $\Set{\beta_1}$ &
		\\[4pt] $\emptyset$ &   $\emptyset$ & & \texttt{fetch ($\beta_2$),} & $\emptyset$ &  $\Set{\beta_2,\beta_3}$%
		\\ $\emptyset$ &   $\emptyset$ & & \texttt{t2 = t2 + 1,} & $\emptyset$ &  $\Set{\beta_2,\beta_3}$%
		\\ $\emptyset$ &   $\emptyset$ & & \texttt{fetch ($\beta_3$),} & $\emptyset$ &  $\Set{\beta_2,\beta_3}$%
		\\ $\emptyset$ &   $\emptyset$ & & \texttt{\{i\} = t2;} & $\emptyset$ &  $\Set{\beta_2,\beta_3}$ &
	\end{tabular} 
	& \hspace{2em} &  \hspace{2em} & 
	\begin{tabular}{ccl||lccc}
		\\ $\emptyset$ & $\Set{\alpha_1}$ & \texttt{fetch ($\alpha_1$),} & & $\emptyset$ &  $\Set{\beta_1}$%
		\\ $\emptyset$ & $\Set{\alpha_1}$ & \texttt{t1 = \{i\},} & & $\emptyset$ &  $\Set{\beta_1}$%
		\\ $\emptyset$ & $\Set{\alpha_2, \alpha_3}$ & \texttt{fetch  ($\alpha_2$),} & & $\emptyset$ &  $\Set{\beta_1}$%
		\\ $\emptyset$ & $\Set{\alpha_2, \alpha_3}$ & \texttt{t1 = t1 + 1;} & & $\emptyset$ &  $\Set{\beta_1}$%
		\\[4pt] $\emptyset$ &  $\Set{\alpha_2, \alpha_3}$ & & \texttt{fetch ($\beta_1$),} & $\emptyset$ &  $\Set{\beta_1}$%
		\\ $\emptyset$ &  $\Set{\alpha_2, \alpha_3}$ & & \texttt{t2 = \{i\};} & $\emptyset$ &  $\Set{\beta_1}$ &
		\\[4pt] $\emptyset$ & $\Set{\alpha_2, \alpha_3}$ & \texttt{fetch ($\alpha_3$),} & & $\emptyset$ &  $\Set{\beta_2,\beta_3}$%
		\\ $\emptyset$ & $\Set{\alpha_2, \alpha_3}$ & \texttt{\{i\} = t1;} & & $\emptyset$ &  $\Set{\beta_2,\beta_3}$%
		\\[4pt] $\emptyset$ & $\emptyset$ & & \texttt{fetch ($\beta_2$),} & $\emptyset$ &  $\Set{\beta_2,\beta_3}$%
		\\ $\emptyset$ &  $\emptyset$ & & \texttt{t2 = t2 + 1,} & $\emptyset$ &  $\Set{\beta_2,\beta_3}$%
		\\ $\emptyset$ &  $\emptyset$ & & \texttt{fetch ($\beta_3$),} & $\emptyset$ &  $\Set{\beta_2,\beta_3}$%
		\\ $\emptyset$ &  $\emptyset$ & & \texttt{\{i\} = t2;} & $\emptyset$ &  $\Set{\beta_2,\beta_3}$ &
	\end{tabular}  
	& \hspace{2em} &  \hspace{2em} & 
	\begin{tabular}{ccl||lccc}
		\\ $\emptyset$ & $\Set{\alpha_1}$ & \texttt{fetch ($\alpha_1$),} & & $\emptyset$ &  $\Set{\beta_1}$%
		\\ $\emptyset$ & $\Set{\alpha_1}$ & \texttt{t1 = \{i\},} & & $\emptyset$ &  $\Set{\beta_1}$%
		\\ $\emptyset$ & $\Set{\alpha_2, \alpha_3}$ & \texttt{fetch  ($\alpha_2$),} & & $\emptyset$ &  $\Set{\beta_1}$%
		\\ $\emptyset$ & $\Set{\alpha_2, \alpha_3}$ & \texttt{t1 = t1 + 1;} & & $\emptyset$ &  $\Set{\beta_1}$%
		\\[4pt] $\emptyset$ &  $\Set{\alpha_2, \alpha_3}$ & & \texttt{fetch ($\beta_1$),} & $\emptyset$ &  $\Set{\beta_1}$%
		\\ $\emptyset$ &  $\Set{\alpha_2, \alpha_3}$ & & \texttt{t2 = \{i\};} & $\emptyset$ &  $\Set{\beta_1}$ &
		\\[4pt] $\emptyset$ &  $\Set{\alpha_2, \alpha_3}$ & & \texttt{fetch ($\beta_2$),} & $\emptyset$ &  $\Set{\beta_2,\beta_3}$%
		\\ $\emptyset$ &  $\Set{\alpha_2, \alpha_3}$ & & \texttt{t2 = t2 + 1,} & $\emptyset$ &  $\Set{\beta_2,\beta_3}$%
		\\ $\emptyset$ &  $\Set{\alpha_2, \alpha_3}$ & & \texttt{fetch ($\beta_3$),} & $\emptyset$ &  $\Set{\beta_2,\beta_3}$%
		\\ $\emptyset$ &  $\Set{\alpha_2, \alpha_3}$ & & \texttt{\{i\} = t2;} & $\emptyset$ &  $\Set{\beta_2,\beta_3}$ &
		\\[4pt] $\emptyset$ & $\Set{\alpha_2, \alpha_3}$ & \texttt{fetch ($\alpha_3$),} & & $\emptyset$ &  $\emptyset$%
		\\ $\emptyset$ & $\Set{\alpha_2, \alpha_3}$ & \texttt{\{i\} = t1;} & & $\emptyset$ &  $\emptyset$&
	\end{tabular}  
\\ \hspace{3em} \\
	\begin{tabular}{ccl||lccc}
		\\ $\emptyset$ 		& $\Set{\alpha_1}$ 	& & \texttt{fetch ($\beta_1$),} 			& $\emptyset$ &  $\Set{\beta_1}$%
		\\ $\emptyset$ 		& $\Set{\alpha_1}$	& & \texttt{t2 = \{i\};} 					& $\emptyset$ &  $\Set{\beta_1}$ &
		\\[4pt] $\emptyset$ & $\Set{\alpha_1}$	& & \texttt{fetch ($\beta_2$),} 			& $\emptyset$ &  $\Set{\beta_2,\beta_3}$%
		\\ $\emptyset$ 		& $\Set{\alpha_1}$	& & \texttt{t2 = t2 + 1,} 					& $\emptyset$ &  $\Set{\beta_2,\beta_3}$%
		\\ $\emptyset$ 		& $\Set{\alpha_1}$	& & \texttt{fetch ($\beta_3$),} 			& $\emptyset$ &  $\Set{\beta_2,\beta_3}$%
		\\ $\emptyset$ 		& $\Set{\alpha_1}$ 	& & \texttt{\{i\} = t2;}					& $\emptyset$ &  $\Set{\beta_2,\beta_3}$ &
		\\[4pt] $\emptyset$ & $\Set{\alpha_1}$ 	& \texttt{fetch ($\alpha_1$),} & 			& $\emptyset$ &  $\emptyset$&
		\\ $\emptyset$ 		& $\Set{\alpha_1}$ 	& \texttt{t1 = \{i\},} & 					& $\emptyset$ &  $\emptyset$&
		\\ $\emptyset$ 		& $\Set{\alpha_2, \alpha_3}$ & \texttt{fetch  ($\alpha_2$),} & 	& $\emptyset$ &  $\emptyset$&
		\\ $\emptyset$ 		& $\Set{\alpha_2, \alpha_3}$ & \texttt{t1 = t1 + 1;} & 			& $\emptyset$ &  $\emptyset$&
		\\[4pt] $\emptyset$ & $\Set{\alpha_2, \alpha_3}$ & \texttt{fetch ($\alpha_3$),} & 	& $\emptyset$ &  $\emptyset$&
		\\ $\emptyset$ 		& $\Set{\alpha_2, \alpha_3}$ & \texttt{\{i\} = t1;} & 			& $\emptyset$ &  $\emptyset$&
	\end{tabular} 
	& \hspace{2em} &  \hspace{2em} & 
	\begin{tabular}{ccl||lccc}
		\\ $\emptyset$ 		& $\Set{\alpha_1}$ 	& & \texttt{fetch ($\beta_1$),} 			& $\emptyset$ &  $\Set{\beta_1}$%
		\\ $\emptyset$ 		& $\Set{\alpha_1}$	& & \texttt{t2 = \{i\};} 					& $\emptyset$ &  $\Set{\beta_1}$ &
		\\[4pt] $\emptyset$ & $\Set{\alpha_1}$ 	& \texttt{fetch ($\alpha_1$),} & 			& $\emptyset$ &  $\Set{\beta_2,\beta_3}$%
		\\ $\emptyset$ 		& $\Set{\alpha_1}$ 	& \texttt{t1 = \{i\},} & 					& $\emptyset$ &  $\Set{\beta_2,\beta_3}$%
		\\ $\emptyset$ 		& $\Set{\alpha_2, \alpha_3}$ & \texttt{fetch  ($\alpha_2$),} & 	& $\emptyset$ &  $\Set{\beta_2,\beta_3}$%
		\\ $\emptyset$ 		& $\Set{\alpha_2, \alpha_3}$ & \texttt{t1 = t1 + 1;} & 			& $\emptyset$ &  $\Set{\beta_2,\beta_3}$%
		\\[4pt] $\emptyset$ & $\Set{\alpha_2, \alpha_3}$ & & \texttt{fetch ($\beta_2$),} 	& $\emptyset$ &  $\Set{\beta_2,\beta_3}$%
		\\ $\emptyset$ 		& $\Set{\alpha_2, \alpha_3}$ & & \texttt{t2 = t2 + 1,} 			& $\emptyset$ &  $\Set{\beta_2,\beta_3}$%
		\\ $\emptyset$ 		& $\Set{\alpha_2, \alpha_3}$ & & \texttt{fetch ($\beta_3$),} 	& $\emptyset$ &  $\Set{\beta_2,\beta_3}$%
		\\ $\emptyset$ 		& $\Set{\alpha_2, \alpha_3}$ & & \texttt{\{i\} = t2;}			& $\emptyset$ &  $\Set{\beta_2,\beta_3}$ &
		\\[4pt] $\emptyset$ & $\Set{\alpha_2, \alpha_3}$ & \texttt{fetch ($\alpha_3$),} & 	& $\emptyset$ &  $\emptyset$&
		\\ $\emptyset$ 		& $\Set{\alpha_2, \alpha_3}$ & \texttt{\{i\} = t1;} & 			& $\emptyset$ &  $\emptyset$&
	\end{tabular}  
	& \hspace{2em} &  \hspace{2em} & 
	\begin{tabular}{ccl||lccc}
		\\ $\emptyset$ 		& $\Set{\alpha_1}$ 	& & \texttt{fetch ($\beta_1$),} 			& $\emptyset$ &  $\Set{\beta_1}$%
		\\ $\emptyset$ 		& $\Set{\alpha_1}$	& & \texttt{t2 = \{i\};} 					& $\emptyset$ &  $\Set{\beta_1}$ &
		\\[4pt] $\emptyset$ & $\Set{\alpha_1}$ 	& \texttt{fetch ($\alpha_1$),} & 			& $\emptyset$ &  $\Set{\beta_2,\beta_3}$%
		\\ $\emptyset$ 		& $\Set{\alpha_1}$ 	& \texttt{t1 = \{i\},} & 					& $\emptyset$ &  $\Set{\beta_2,\beta_3}$%
		\\ $\emptyset$ 		& $\Set{\alpha_2, \alpha_3}$ & \texttt{fetch  ($\alpha_2$),} & 	& $\emptyset$ &  $\Set{\beta_2,\beta_3}$%
		\\ $\emptyset$ 		& $\Set{\alpha_2, \alpha_3}$ & \texttt{t1 = t1 + 1;} & 			& $\emptyset$ &  $\Set{\beta_2,\beta_3}$%
		\\[4pt] $\emptyset$ & $\Set{\alpha_2, \alpha_3}$ & \texttt{fetch ($\alpha_3$),} & 	& $\emptyset$ &  $\Set{\beta_2,\beta_3}$%
		\\ $\emptyset$ 		& $\Set{\alpha_2, \alpha_3}$ & \texttt{\{i\} = t1;} & 			& $\emptyset$ &  $\Set{\beta_2,\beta_3}$%
		\\[4pt] $\emptyset$ & $\emptyset$	& & \texttt{fetch ($\beta_2$),} 			& $\emptyset$ &  $\Set{\beta_2,\beta_3}$%
		\\ $\emptyset$ 		& $\emptyset$	& & \texttt{t2 = t2 + 1,} 					& $\emptyset$ &  $\Set{\beta_2,\beta_3}$%
		\\ $\emptyset$ 		& $\emptyset$	& & \texttt{fetch ($\beta_3$),} 			& $\emptyset$ &  $\Set{\beta_2,\beta_3}$%
		\\ $\emptyset$ 		& $\emptyset$ 	& & \texttt{\{i\} = t2;}					& $\emptyset$ &  $\Set{\beta_2,\beta_3}$ &
	\end{tabular}  
\end{tabular}
}
\end{center}

\begin{itemize} 
	\item By the order reduction theorem, this program will not behave differently under preemptive scheduler
	\item Number of interleavings reduced by a factor of 150 for this toy program
\end{itemize}
\end{frame}

\begin{frame}
Step 2 to a better life: \textbf{ownership deduction}

No ownership annotations?
\begin{center}
	\begin{tabular}{ccl||lccc}
		$\emptyset$ & $\Set{\alpha_1}$ & \texttt{fetch ($\alpha_1$),} & & $\emptyset$ &  $\Set{\beta_1}$%
		\\ $\emptyset$ & $\Set{\alpha_1}$ & \texttt{t1 = \{i\},} & & $\emptyset$ &  $\Set{\beta_1}$%
	\end{tabular}  
\end{center}
What to acquire now?
\end{frame} 

\begin{frame} 
Execution of one thread between two LPs does not depend on other threads
\begin{center}
	\begin{tabular}{ccl||lccc}
		$\emptyset$ & $\Set{\alpha_1}$ & \texttt{fetch ($\alpha_1$),} & & $\emptyset$ &  $\Set{\beta_1}$%
		\\ $\emptyset$ & $\Set{\alpha_1}$ & \texttt{t1 = \{i\}{\color{red},}} & & $\emptyset$ &  $\Set{\beta_1}$%
		\\ $\emptyset$ & $\Set{\alpha_2, \alpha_3}$ & \color{red}\texttt{fetch  ($\alpha_2$),} & & $\emptyset$ &  $\Set{\beta_1}$%
		\\ $\emptyset$ & $\Set{\alpha_2, \alpha_3}$ & \color{red}\texttt{t1 = t1 + 1;} & & $\emptyset$ &  $\Set{\beta_1}$%
		\\[4pt] $\emptyset$ &  $\Set{\alpha_2, \alpha_3}$ & & \texttt{fetch ($\beta_1$),} & $\emptyset$ &  $\Set{\beta_1}$%
		\\ $\emptyset$ &  $\Set{\alpha_2, \alpha_3}$ & & \texttt{t2 = \{i\};} & $\emptyset$ &  $\Set{\beta_1}$ &
		\\[4pt] $\emptyset$ &  $\Set{\alpha_2, \alpha_3}$ & & \texttt{fetch ($\beta_2$),} & $\emptyset$ &  $\Set{\beta_2,\beta_3}$%
		\\ $\emptyset$ &  $\Set{\alpha_2, \alpha_3}$ & & \texttt{t2 = t2 + 1,} & $\emptyset$ &  $\Set{\beta_2,\beta_3}$%
		\\ $\emptyset$ &  $\Set{\alpha_2, \alpha_3}$ & & \texttt{fetch ($\beta_3$),} & $\emptyset$ &  $\Set{\beta_2,\beta_3}$%
		\\ $\emptyset$ &  $\Set{\alpha_2, \alpha_3}$ & & \texttt{\{i\} = t2;} & $\emptyset$ &  $\Set{\beta_2,\beta_3}$ &
		\\[4pt] $\emptyset$ & $\Set{\alpha_2, \alpha_3}$ & \color{red}\texttt{fetch ($\alpha_3$),} & & $\emptyset$ &  $\emptyset$%
		\\ $\emptyset$ & $\Set{\alpha_2, \alpha_3}$ & \texttt{{\color{red}\{i\} = t1};} & & $\emptyset$ &  $\emptyset$&
	\end{tabular}  
\end{center}
But required ownership only depends on execution until next LP
\end{frame} 

\begin{frame}
First idea: execute until next LP
\begin{center}
	\begin{tabular}{ccl||lccc}
		$\emptyset$ & $\Set{\alpha_1}$ & \texttt{fetch ($\alpha_1$),} & & $\emptyset$ &  $\Set{\beta_1}$%
		\\ $\emptyset$ & $\Set{\alpha_1}$ & \texttt{t1 = \{i\}{\color{red},}} & & $\emptyset$ &  $\Set{\beta_1}$%
		\\ $\emptyset$ & $\Set{\alpha_2, \alpha_3}$ & \color{red}\texttt{fetch  ($\alpha_2$),} & & $\emptyset$ &  $\Set{\beta_1}$%
		\\ $\emptyset$ & $\Set{\alpha_2, \alpha_3}$ & \color{red}\texttt{t1 = t1 + 1;} & & $\emptyset$ &  $\Set{\beta_1}$%
		\\[4pt] $\emptyset$ & $\Set{\alpha_2, \alpha_3}$ & \color{red}\texttt{fetch ($\alpha_3$),} & & $\emptyset$ &  $\emptyset$%
		\\ $\emptyset$ & $\Set{\alpha_2, \alpha_3}$ & \texttt{{\color{red}\{i\} = t1};} & & $\emptyset$ &  $\emptyset$&
	\end{tabular}  
\end{center}

Problems: 
\begin{enumerate} 
	\item maybe no more LP?
	\item slow
\end{enumerate}
\end{frame}

\begin{frame}
Better idea: remember accesses by yourself and by others
\begin{enumerate} 
	\item Accesses by others must not be owned by you (upper bound)
	\item Local accesses by yourself must be owned by you (lower bound)
\end{enumerate} 
Construct lower and upper bound for ownerships for each thread $X$
\begin{align*}
	O_\mathit{min}(X)\  &\subseteq \ O(X) \ \subseteq \ O_\mathit{max}(X) 
	\\
	P_\mathit{min}(X) \ &\subseteq \ P(X) \ \subseteq \ P_\mathit{max}(X)
\end{align*}
\end{frame} 
\begin{frame}
Example (for P of Thread 1)
\begin{center}
	\begin{tabular}{ccl||l}
		$P_{min}$ & $P_{max}$ 
		\\ 
			$\Set{\alpha_1}$ 			& $\mathcal A$ 	& \texttt{fetch ($\alpha_1$),} & \\ 
			$\emptyset$		 			& $\mathcal A$ 	&  \texttt{t1 = \{i\},} & \\ 
			$\Set{\alpha_2}$		 	& $\mathcal A$ 	&  \texttt{fetch  ($\alpha_2$),} \\ 
			$\Set{\alpha_2}$		 	& $\mathcal A$ 	&  \texttt{t1 = t1 + 1;} 
		\\[4pt]
			$\Set{\alpha_2}$		 	& $\mathcal A$ 	& & \texttt{fetch ($\beta_1$),}  \\
			$\Set{\alpha_2}$		 	& $\mathcal A$ 	& & \texttt{t2 = \{i\};}
		\\[4pt] 
			$\Set{\alpha_2}$		 	& $\mathcal A$ 	& & \texttt{fetch ($\beta_2$),} \\ 
			$\Set{\alpha_2}$		 	& $\mathcal A$ 	& & \texttt{t2 = t2 + 1,} \\
			$\Set{\alpha_2}$		 	& $\mathcal A$ 	& & \texttt{fetch ($\beta_3$),} \\
			$\Set{\alpha_2}$		 	& $\mathcal A \setminus \Set{\texttt{i}}$ & & \texttt{\{i\} = t2;} 
		\\[4pt] 
			$\Set{\alpha_2, \alpha_3}$	& $\mathcal A \setminus \Set{\texttt{i}}$ & \texttt{fetch ($\alpha_3$),} & \\ 
			$\emptyset$		 	& $\mathcal A$ 		& \texttt{\{i\} = t1;} & 
	\end{tabular}
\end{center}

\end{frame}

\begin{frame}
Example (for O of Thread 2)
\begin{center}
	\scalebox{0.85}{
		\begin{tabular}{l||lcc}
			&&$O_{min}$ & $O_{max}$ 
			\\ 
			\texttt{fetch ($\alpha_1$),} & & $\emptyset$ & $ \mathcal{A} \setminus \Set{\alpha_1}$ \\ 
			\texttt{t1 = \{i\},} & & $\emptyset$ & $ \mathcal{A} \setminus \Set{\alpha_1, \texttt{i}}$  \\ 
			\texttt{fetch  ($\alpha_2$),} & & $\emptyset$ & $ \mathcal{A} \setminus \Set{\alpha_1, \alpha_2, \texttt{i}}$\\ 
			\texttt{t1 = t1 + 1;} & & $\emptyset$ & $ \mathcal{A} \setminus \Set{\alpha_1, \alpha_2, \texttt{i}}$ 
			\\[4pt]
			& \texttt{fetch ($\beta_1$),} & $\emptyset$ & $ \mathcal{A} \setminus \Set{\alpha_1, \alpha_2, \texttt{i}}$  \\
			& \texttt{t2 = \{i\};} & $\emptyset$ & $ \mathcal{A}$
			\\[4pt] 
			& \texttt{fetch ($\beta_2$),} & $\emptyset$ & $ \mathcal{A}$ \\ 
			& \texttt{t2 = t2 + 1,} & $\emptyset$ & $ \mathcal{A}$ \\
			& \texttt{fetch ($\beta_3$),} & $\emptyset$ & $ \mathcal{A}$ \\
			& \texttt{\{i\} = t2;}  & $\emptyset$ & $ \mathcal{A}$
			\\[4pt] 
			\texttt{fetch ($\alpha_3$),} & & $\emptyset$ & $ \mathcal{A} \setminus \Set{\alpha_3}$\\ 
			\texttt{\{i\} = t1;} & & $\emptyset$ & $ \mathcal{A} \setminus \Set{\alpha_3, \texttt{i}}$
	\end{tabular}}
\end{center}
\end{frame}

\begin{frame} 
\begin{itemize} 
	\item Do this for O,P for each thread in all cooperative schedules
	\item Theorem: iff for all threads $X$ before each LP and at end of schedule we have 
	\begin{align*}
	O_\mathit{min}(X)\  & \subseteq \ O_\mathit{max}(X) 
	\\
	P_\mathit{min}(X) \ & \subseteq \ P_\mathit{max}(X)
	\end{align*}
	then the program satisfies ownership conditions
	\item No need for ownership annotation if we look at interleavings
\end{itemize}
\end{frame} 

\begin{frame}
Step 3 to a better life: \textbf{order reduction+\onslide<2->{invariants}}

\onslide<3->
\begin{itemize}
	\item Purely sequential reasoning with invariants from IP to IP
	\item Global invariant $I_{Sh}$, argues about everything that can only be changed by LP
	\item Local invariants $I_{L,X}$ for each thread $X$, argues about everything that can only be changed by thread $X$
\end{itemize}

\begin{center}
	\includefig{MLayoutInvariants}
\end{center}
\end{frame}


\begin{frame}
Prove that this program increments at most once

\onslide<2->{Ownership: $\texttt{x} \in P(1), \quad \texttt{y} \in P(2)$}
\begin{center}
	\texttt{
	\begin{tabular}{ll||l}
	\color{gray} 1&\{x\} = 1; & \{y\} = 1; \\
	\color{gray} 2& t1 = \{y\}, \onslide<2->{acq z if !y} & t2 = \{x\}, \onslide<2->{acq z if !x} \\
	\color{gray} 3&if (! t1) & if (! t2) \\
	\color{gray} 4&\quad t1 = z, & \quad t2 = z, \\
	\color{gray} 5&\quad z = t1+1; & \quad z = t2+1;
	\end{tabular}}
\end{center}
\onslide<3-> Invariants
\begin{align*}
	\texttt{z} \not\in O(1) \ \land \ \texttt{z} \not\in O(2) \quad &\to \quad \texttt{z} = 0 & \onslide<4->{\text{(Sh)}}
	\\
	\texttt{x} = 0 \quad &\to\quad \texttt{z} \not\in O(1) & \onslide<5->{\text{(Sh)}}
	\\
	\texttt{y} = 0 \quad &\to\quad \texttt{z} \not\in O(2) & \onslide<5->{\text{(Sh)}}
	\\
	\texttt{pc1} = 2 \quad &\to\quad \texttt{x} = 1 \land \texttt{z} \not\in O(1) & \onslide<6->{\text{(L,1)}}
	\\
	\texttt{pc2} = 2 \quad &\to\quad \texttt{y} = 1 \land \texttt{z} \not\in O(2) & \onslide<6->{\text{(L,2)}}
	\\
	\texttt{z} \in O(1) \quad &\to\quad \texttt{z} \le 1 & \onslide<7->{\text{(L,1)}}
	\\
	\texttt{z} \in O(2) \quad &\to\quad \texttt{z} \le 1 & \onslide<7->{\text{(L,2)}}
\end{align*}
\end{frame}

\begin{frame}
\begin{center}
	\texttt{
		\begin{tabular}{ll}
			\color{gray} 1&\{x\} = 1; \\
		\end{tabular}}
\end{center}

\begin{align*}
\texttt{z} \not\in O(1) \ \land \ \texttt{z} \not\in O(2) \quad &\to \quad \texttt{z} = 0
\\
\texttt{x} = 0 \quad &\to\quad \texttt{z} \not\in O(1)
\\
\texttt{y} = 0 \quad &\to\quad \texttt{z} \not\in O(2)
\\
\texttt{pc1} = 2 \quad &\to\quad \texttt{x} = 1 \land \texttt{z} \not\in O(1) 
\\
\texttt{z} \in O(1) \quad &\to\quad \texttt{z} \le 1
\end{align*}
\end{frame}

\begin{frame}
\begin{center}
	\texttt{
	\begin{tabular}{ll}
		\color{gray} 2& t1 = \{y\}, acq z if !y  \\
		\color{gray} 3&if (! t1)  \\
		\color{gray} 4&\quad t1 = z,   \\
		\color{gray} 5&\quad z = t1+1; 
	\end{tabular}}
\end{center}
\scalebox{0.5}{.}
\begin{align*}
\texttt{z} \not\in O(1) \ \land \ \texttt{z} \not\in O(2) \quad &\to \quad \texttt{z} = 0  & \onslide<4>{*}
\\
\texttt{x} = 0 \quad &\to\quad \texttt{z} \not\in O(1)
\\
\texttt{y} = 0 \quad &\to\quad \texttt{z} \not\in O(2) & \onslide<3>{*}
\\
\texttt{pc1} = 2 \quad &\to\quad \texttt{x} = 1 \land \texttt{z} \not\in O(1)   & \onslide<2>{*}
\\
\texttt{z} \in O(1) \quad &\to\quad \texttt{z} \le 1
\end{align*}
\begin{description} 
	\item[Case 1:] $\texttt{y}$ \\
		Nothing changes
	\item[Case 2:] $\texttt{!y}$
\[ \onslide<2->{\texttt{z} \not \in O(1)}\onslide<3->{,\ \texttt{z} \not\in O(2)}\onslide<4->{,\ \texttt{z} = 0} \]
\end{description}
\end{frame}

\begin{frame}
\textbf{Summary}

Two ways of reasoning: 
\begin{description} 
	\item[Cooperative scheduler]:
	
	\begin{itemize}
		\item ownership automatically deduced
		\item very low number of interleavings, probably still too many for big programs
	\end{itemize} 
	\item<2->[Sequential modular reasoning]:
	
	\begin{itemize} 
		\item Local+Shared Invariants
		\item ownership manual
		\item no interleavings, instead 
		
			\begin{itemize} 
				\item shared invariants that other threads obey
				\item information about shared state is lost at each IP
			\end{itemize}
	\end{itemize}
\end{description}
\end{frame}

\begin{frame}
\textbf{But what about HW interrupts?}

\begin{itemize}
	\item In cooperative scheduler, interrupts only occur at IP
	\begin{center} 
	\includefig{BlockSuspensionComplete}	
\end{center} 
	\item<2-> In preemptive scheduler, interrupts occur at any point
	\begin{center} 
		\includefig{BlockSuspension}	
	\end{center} 
\end{itemize}

\end{frame}


\begin{frame} 
Things evil IH can do: \textbf{mess up thread}

\begin{center}\texttt{
		\begin{tabular}{l||l}
			T & IH \\
			t1 = \{x\}, & t1 = 1, \\
			if (t1) & \{eret\}; \\
			\qquad t1 = 2; 
		\end{tabular}
}\end{center}

In cooperative scheduler:
\begin{center}\texttt{
		\begin{tabular}{ccc}
			\begin{tabular}{l||l}
				t1 = \{x\}, 	
				\\ 
				if (t1) 	
				\\
				// nope \\
				& t1 = 1, \\
				& \{eret\};
			\end{tabular}
			& \hspace{2em}
			&
			\begin{tabular}{l||l}
				& t1 = 1, \\
				& \{eret\}; \\
				t1 = \{x\}, \\
				if (t1) 	
				\\
				// nope
			\end{tabular}
		\end{tabular}
	}
\end{center}
	\[ \texttt{t1} \in \Set{ 0, 1 } \]
\end{frame}
 
\begin{frame}
In preemptive scheduler: 
\begin{center}\texttt{
	\begin{tabular}{l||l}
		t1 = \{x\}, 	
		\\ 
		& t1 = 1, \\
		& \{eret\}; \\
		if (t1) \\
		\qquad t1 = 2; 
	\end{tabular}
	}
\end{center}
\end{frame} 

\begin{frame} 
Things evil IH can do: \textbf{read out state}

\begin{center}\texttt{
		\begin{tabular}{l||l}
			T & IH \\
			t1 = 1, 	& if (t1) \\
			t2 = \{x\}; 	& \qquad x=1, \\
						& \{eret\};
		\end{tabular}
}\end{center}

In cooperative scheduler:
\begin{center}\texttt{
\begin{tabular}{ccc}
	\begin{tabular}{l||l}
			t1 = 1, 	
			\\ 
			t2 = \{x\}; 	
			\\
			& if (t1) \\
			& \qquad x=1, \\
			& \{eret\};
	\end{tabular}
& \hspace{2em}
&
	\begin{tabular}{l||l}
	& if (t1) \\
	& // nope \\
	& \{eret\}; \\
	t1 = 1, \\ 
	t2 = \{x\};
\end{tabular}
\end{tabular}
}
\end{center}
\[ \texttt{t2} = 0 \]
\end{frame} 
\begin{frame}
In preemptive scheduler:
\begin{center}
	\texttt{
		\begin{tabular}{l||l}
		t1 = 1, \\ 
		& if (t1) \\
		& \qquad x = 1, \\
		& \{eret\}; \\
		t2 = \{x\};
	\end{tabular}
}
\end{center} 

\end{frame} 

\begin{frame} 
Obvious conditions
\begin{enumerate} 
	\item don't depend on context
	\item restore state after eret
\end{enumerate} 
\end{frame}

\begin{frame}
Typical interrupt service routine (ISR):
\begin{center}\texttt{
	\begin{tabular}{l}
		// \{jisr\}, \\
		store context; \\
		\{enable interrupts\}; \\
		... \\
		\{disable interrupts\}; \\
		restore context, \\
		\{eret\};
	\end{tabular}
}\end{center}

\begin{enumerate} 
	\item obviously uses state (copies it to memory)
	\item not so obvious: can't fully restore state
\end{enumerate} 
\end{frame}

\begin{frame}
Semantics of \texttt{eret}:
\begin{align*}
	pc' &= epc \\
	mode' &= emode \\
	mask' &= emask 
\end{align*} 

... and who restores $epc$, $emode$, $emask$?
\end{frame} 

\begin{frame}
Partition into four classes of registers:

\alt<2->
{\alt<3->
{\begin{center} 
		\includefig{MLayoutPRNonPre}
\end{center}}
{\begin{center} 
	\includefig{MLayoutPRPre}
\end{center}}}
{\begin{center} 
	\includefig{MLayoutPR}
\end{center}}
\end{frame} 

\begin{frame} 
Keep track of context with ``dirtiness'' formalism:

\[ d(a) = \begin{cases} \bot & \text{not dirty} \\
	\square & \text{contains trash} \\
	(r,X) & \text{contains value of register $r$ of thread $X$}
 	\end{cases} \]

So to restore context of thread $X$:
\[ eret \ \land \ d(epc) = (pc,X) \ \land \ d(emode) = (mode,X) \ldots \]
\end{frame} 



\begin{frame} 
Summary of Dirtiness Conditions:
\begin{enumerate} 
	\item moving dirty data around is allowed, including into memory or other registers 
	\item using dirty data for addresses (e.g., by writing it into PTE) and some other sensitive stuff is forbidden
	\item on eret, restorable registers must be restored by having 
	\[ d(epc) = (pc,X) \ \land \ \ldots \]
\end{enumerate}
\end{frame} 
\bibliography{bibliography}



\end{document}
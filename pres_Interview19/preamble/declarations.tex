%\theoremstyle{definition}
%\newtheorem{definition}{Definition}
%\newtheorem{lemma}{Lemma}
%\newtheorem{theorem}{Theorem}

% note: the next 2 environments are used only in sysbook; have to be adapted wrt to the style reference
\newtheorem{invariant}{Invariant}
\newtheorem{ccond}{Context Condition}

% make float placement on top the default:
\floatplacement{figure}{t}
\floatplacement{table}{t}
\renewcommand{\floatpagefraction}{0.8}

% tikz
\tikzset{timing/slope=0.1}
\tikzset{timing/.style = {y=2.5ex, x = 4.0ex}}

% other
\graphicspath{{figures/}}
\bibliographystyle{alpha}

% remove later
\newcommand{\typo}[1]{\textcolor{blue}{\textbf{#1}}}
\newcommand{\bbox}[1]{\fcolorbox{blue}{white}{#1}}

%%%%%%%%%%%%%%%%%%%%%%%%%%%%%%%%%%%%%%%%%%%%%%%%%%%%%%%%%%%%%%%%
% Define your desired shortcuts here:
\mathchardef\hy="2D
\newcommand{\cm}{\mbox{,}}
\newcommand{\setB}{\mathbb{B}}
\newcommand{\setN}{\mathbb{N}}
\newcommand{\setZ}{\mathbb{Z}}
\newcommand{\setR}{\mathbb{R}}
\newcommand{\setS}{\mathbb{S}}
\newcommand{\B}{\mathbb{B}}
\newcommand{\N}{\mathbb{N}}
\newcommand{\R}{\mathbb{R}}
\newcommand{\nat}{\mathbb{N}}
\newcommand{\Z}{\mathbb{Z}}
\newcommand{\NOT}[1]{\overline{#1}}
\newcommand{\AND}{\land}
\newcommand{\OR}{\lor}
\newcommand{\Sum}{\sum\limits}
\newcommand{\tmod}{\mathrel{\mathrm{tmod}}}
\newcommand{\pskip}{\smallskip\par\noindent}
\newcommand{\ignore}[1]{\relax}
\newcommand{\figscale}{0.834}
%\renewcommand{\pskip}{\smallskip\bbox{\mbox{\quad}}\par\noindent}
\newcommand{\smalltt}[1]{\text{\small\texttt{#1}}}
\newcommand{\super}[1]{\textsuperscript{#1}}
\newcommand{\sub}[1]{\textsubscript{#1}}
\newcommand{\langlett}{{\fontfamily{cmbr}\selectfont\textlangle}}
\newcommand{\ranglett}{{\fontfamily{cmbr}\selectfont\textrangle}}
\renewcommand{\theFancyVerbLine}{\text{\small\arabic{FancyVerbLine}:}}
\renewenvironment{verbatim}[0]{\Verbatim}{\endVerbatim}

% figures
\newcommand{\tfig}[3]{%
\centering%
\adjustbox{scale = \figscale}{\input{figures/#1.pdf_t}}%
\caption{#3}\label{fig:#2}%
}

\newcommand{\tsubfigV}[3]{%
\subcaptionbox{\label{#2} #3}{\import{figures/}{#1.pdf_tex}}%
}

\newcommand{\includefig}[1]{\import{figures/}{#1.pdf_tex}}


\newcommand{\tschemfig}[4]{%
\figure[tp]%
\addtolength{\subfigcapskip}{0.1in}%
\centering%
\subfigure[\label{fig:#1} symbol]{\trimbox{-0.5\textwidth+0.5\width 0pt 0pt}{\raisebox{#3mm}{\adjustbox{scale = \figscale}{\input{figures/#1.pdf_t}}}}}\\%
\subfigure[\label{fig:#1impl} implementation]{\trimbox{-0.5\textwidth+0.5\width 0pt 0pt}{\raisebox{#4mm}{\adjustbox{scale = \figscale}{\input{figures/#1impl.pdf_t}}}}}%
\caption{#2}\label{fig:#1-all}%
\endfigure%
}

%note: command "remark" is already defined in the style file
%\newcommand{\remark}[1]{\marginpar{\framebox{\parbox[t]{16mm}{
  %\tiny\raggedright #1}}}}
%\newcommand{\oldremark}[1]{\relax}

% code
\fvset{
	frame=lines,
	framerule=0.2pt,
	framesep=4pt,
	resetmargins=true,
	xleftmargin=16pt,
	xrightmargin=16pt,
%	numbers=left,
%	numberblanklines=false,
%	numbersep=-16pt,
	tabsize=0,
	fontfamily=courier,
	fontsize=\small,
	commandchars=\\\@\#
}

\lstnewenvironment{code}
    {\lstset{}%
      \csname lst@SetFirstLabel\endcsname}
    {\csname lst@SaveFirstLabel\endcsname}
    \lstset{
      language=Haskell,
      mathescape = true,
      commentstyle= \sffamily\itshape,
      basicstyle=\small\ttfamily,
      keywordstyle=\color{DarkBlue}\bfseries,
      flexiblecolumns=false,
      basewidth={0.5em,0.45em},
      morecomment=[l]{//},
      numbers = left,
      numberstyle = \color{Gray}\tiny\ttfamily,
      breaklines = true,
      literate={+}{{$+$}}1 {*}{{$*$}}1 {=}{{$=$}}1 {!=}{{$\neq$}}2
               {>}{{$>$}}1 {<}{{$<$}}1 {\\}{{$\lambda$}}1
               {->}{{$\rightarrow$}}2 {>=}{{$\geq$}}2 {<-}{{$\leftarrow$}}2
               {<=}{{$\leq$}}2 {=>}{{$\Rightarrow$}}2
               {>>}{{$\gg$}}1
               {<<}{{$\ll$}}1
               {|}{{$\mid$}}1
               {||}{{$\vee$}}2 {&&}{{$\wedge$}}2
               {@}{{$\circ$}}1
               {?}{{{\color{DarkBlue}{?}}}}1
               {...}{{$\ldots$}}3
    }

% forbids footnotes to travel
%\interfootnotelinepenalty=1000000000
\interfootnotelinepenalty=10000

% theorems of amsthm
